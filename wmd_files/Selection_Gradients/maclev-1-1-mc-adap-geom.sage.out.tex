\documentclass{article}
\usepackage{amsmath}
\usepackage{amssymb}
\usepackage{amsfonts}
\usepackage{graphicx}
\pagestyle{empty}
\usepackage[utf8]{inputenc}
\usepackage[T1]{fontenc}
\usepackage{latexml}
\oddsidemargin 0.0in
\evensidemargin 0.0in
\textwidth 6.45in
\topmargin 0.0in
\headheight 0.0in
\headsep 0.0in
\textheight 9.0in
\usepackage{amsmath}
\newcommand{\ZZ}{\Bold{Z}}
\newcommand{\NN}{\Bold{N}}
\newcommand{\RR}{\Bold{R}}
\newcommand{\CC}{\Bold{C}}
\newcommand{\QQ}{\Bold{Q}}
\newcommand{\QQbar}{\overline{\QQ}}
\newcommand{\GF}[1]{\Bold{F}_{#1}}
\newcommand{\Zp}[1]{\ZZ_{#1}}
\newcommand{\Qp}[1]{\QQ_{#1}}
\newcommand{\Zmod}[1]{\ZZ/#1\ZZ}
\newcommand{\CDF}{\Bold{C}}
\newcommand{\CIF}{\Bold{C}}
\newcommand{\CLF}{\Bold{C}}
\newcommand{\RDF}{\Bold{R}}
\newcommand{\RIF}{\Bold{I} \Bold{R}}
\newcommand{\RLF}{\Bold{R}}
\newcommand{\CFF}{\Bold{CFF}}
\newcommand{\Bold}[1]{\mathbf{#1}}
\begin{document}
With $c_{00}=u_0$, the vector $\mathbf p$ is 
\[
  {\mathbf p} = \left(\begin{array}{c}
  b_{0}\\
  m_{0}\\
  {c}_{00}
\end{array}\right) = \left(\begin{array}{c}
  1\\
  1\\
  u_{0}
\end{array}\right)
\]
And the selection gradient of $\mathbf p$ is 
\[
  \mathbf S({\mathbf p}) = \left(\begin{array}{c}
  -\frac{\hat{X}_{0} {c}_{00}^{2} w_{0} - K_{0} {c}_{00} r_{0} w_{0} + m_{0} r_{0}}{r_{0}}\\
  -b_{0}\\
  -\frac{\hat{X}_{0} b_{0} {c}_{00} w_{0} - K_{0} b_{0} r_{0} w_{0}}{r_{0}}
\end{array}\right) = \left(\begin{array}{c}
  0\\
  -1\\
  -\frac{2 \, u_{0} - 1}{u_{0}} + 2
\end{array}\right)
\]
The derivative of $\mathbf p$ is 
\[
  \frac{d{\mathbf p}}{du_{0}} = \left(\begin{array}{c}
  0\\
  0\\
  1
\end{array}\right), \]
and we can recover the motion of $u_0$: 
\[ \mathbf S(u_{0}) = \frac{d{\mathbf p}}{du_{0}}^\mathrm{T} \mathbf S(\mathbf p(u_{0})) = -\frac{2 \, u_{0} - 1}{u_{0}} + 2 \]

\[
  \frac{du_0}{dt} = \gamma\hat X_0 \mathbf S(u_{0}) = -X_{0} {\left(\frac{2 \, u_{0} - 1}{u_{0}} - 2\right)}
\]
and the motion of $\mathbf p(u_0)$:
\[
  \frac{d\mathbf p(u_0)}{dt} = \gamma\hat X_0 \frac{d\mathbf p}{du_0} \frac{d\mathbf p}{du_0}^\mathrm{T} \mathbf S(\mathbf p(u_{0})) = \left(\begin{array}{c}
  0\\
  0\\
  -X_{0} {\left(\frac{2 \, u_{0} - 1}{u_{0}} - 2\right)}
\end{array}\right)
\]
So now we can get a look at the geometry of ${\mathbf p}$.  At $u_0=\frac{2}{3}$,
\[ {\mathbf p} = \left(\begin{array}{c}
  1\\
  1\\
  \frac{2}{3}
\end{array}\right), \]
\[ \mathbf S({\mathbf p}) = \left(\begin{array}{c}
  0\\
  -1\\
  \frac{3}{2}
\end{array}\right) \]
The constraint curve for $\mathbf p(u_{0})$: is 
\[ \left(\begin{array}{c}
  b\left(u_{0}\right)\\
  m\left(u_{0}\right)\\
  c_{0}\left(u_{0}\right)
\end{array}\right) = \left(\begin{array}{c}
  1\\
  1\\
  u_{0}
\end{array}\right) \]
And on that curve, 
\[ \left.\frac{d\mathbf p(u_{0})}{dt}\right|_{u_0=\frac{2}{3}} = \left(\begin{array}{c}
  0\\
  0\\
  \frac{9}{8}
\end{array}\right) \]

\end{document}
