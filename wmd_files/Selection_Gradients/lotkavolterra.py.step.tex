\begin{lstlisting}[language=Python]
from sage.all import *
 
from dynamicalsystems import *

from sage.symbolic.relation import solve
#from sage.symbolic.function_factory import function

class GeneralizedLotkaVolterraModel(PopulationDynamicsSystem):
    """The GLV model is dX_i/dt = (r_i + sum_j a_i_j X_j) X_i"""
    def __init__(self, x_indices = [], X = indexer('X'), r = indexer('r'),
      a = indexer_2d('a'),
      bindings = Bindings()):
        self._indexers = {'r': r, 'a': a, 'X': X}
        super(GeneralizedLotkaVolterraModel,self).__init__(
            [ ], x_indices, X, bindings=bindings )
    def flow(self):
        return self.make_flow(**self._indexers)
    def make_flow(self, X, r, a):
        return dict( (X[i],
          X[i]*(r[i] + sum( a[i][j]*X[j] for j in self._population_indices )))
          for i in self._population_indices)
    def interior_equilibrium_bindings( self, phenotype_bindings=Bindings() ):
	eq = self.bind( phenotype_bindings ).interior_equilibria()[0]
	eb = Bindings( { vhat:eq[vhat] for vhat in self.equilibrium_vars() } )
	#print 'equilibrium bindings:', eb
	return eb

class LotkaVolterraException(Exception):
    def __init__(self, message):
	self.value = message
    def __str__(self):
	return repr(self.value)

def get_coeff( expr, vars, power ):
    for c, p in expr.coefficients( vars ):
        if p == power:
	    return c

class DerivedLotkaVolterraModel( GeneralizedLotkaVolterraModel ):
    """This is a LotkaVolterraModel made by decomposing the dynamics of
    some other population dynamics model into linear and quadratic terms.
    It has the attributes of a GeneralizedLotkaVolterraModel, plus some
    additional information about the original model."""
    def __init__(self, model, r_name='r', a_name='a', r_indexer=None, a_indexer=None ):
	self._original_model = model
	self._a_name = a_name
	self._r_name = r_name
	if a_indexer is None: a_indexer = indexer_2d(a_name)
	if r_indexer is None: r_indexer = indexer_2d(r_name)
	self._a_indexer = a_indexer
	self._r_indexer = r_indexer
	self.calculate_lv()
        super(DerivedLotkaVolterraModel,self).__init__(
            model._population_indices,
	    X = model._population_indexer,
            r = r_indexer,
            a = a_indexer
	)
    def calculate_lv(self):
	model = self._original_model
        x_indexer = model._population_indexer
        aij_dict = {}
        vars = {v:1 for v in model._vars}
        for i in model._population_indices:
            xi = x_indexer[i]
	    lvi = model._flow[xi].expand()
            print 'Inferring LV coefficients from', xi, 'equation:', lvi
	    xic = get_coeff( lvi, xi, 1 )
	    if xic is None: xic = 0
	    ri = xic
	    aij = get_coeff( lvi, xi, 2 )
	    if aij is None: aij = 0
	    aij_dict[self._a_indexer[i][i]] = aij
            lvi -= aij * xi**2
	    for j in model._population_indices:
	        if j != i:
                    xj = x_indexer[j]
		    ri = get_coeff( ri, xj, 0 )
		    if ri is None: ri = 0
		    aij = get_coeff( xic, xj, 1 )
		    if aij is None: aij = 0
		    aij_dict[self._a_indexer[i][j]] = aij
                    print self._a_indexer[i][j], ':', aij
                    lvi -= aij * xi * xj
            print self._r_indexer[i], ':', ri
	    aij_dict[self._r_indexer[i]] = ri
            lvi -= ri * xi
            if lvi != 0:
                raise LotkaVolterraException( "Population dynamics has excess terms in " + str(i) + "'th component: " + str(lvi) )
            del vars[xi]
        if len(vars) > 0:
            raise LotkaVolterraException( "Population dynamics has extra variables: " + ', '.join(vars.keys()) )
	print 'aij_dict:',aij_dict; sys.stdout.flush()
	self._A_bindings = Bindings( aij_dict ) 
    def set_population_indices(self, npi):
	super(DerivedLotkaVolterraModel, self).set_population_indices(npi)
	self.calculate_lv()
    def mutate(self, resident_index):
	mutant = self._original_model.mutate(resident_index)
	self.set_population_indices( self._original_model._population_indices )
	return mutant

class inner_curry(indexer):
    def __init__(self, ixr, j):
	self._f = ixr
	self. _j = j
    def __getitem__(self, i):
	return self._f[i][self._j]

class backward_curry_indexer_2d( indexer ):
    def __init__(self, ixr):
	self._f = ixr
    def __getitem__(self, j):
	# return a thing that maps i --> ixr[i][j]
	return inner_curry(self._f, j)

class LotkaVolterraAdaptiveDynamics( object ):
    '''Not an AdaptiveDynamicsModel class, but a helper that works with one.'''
    def __init__( self, ad, lv_model=None, r_name='r', a_name='a', r_name_indexer=None, a_name_indexer=None ):
        self._adaptivedynamics = ad

	print 'ad bindings:', self._adaptivedynamics._bindings
	print 'model bindings:', self._adaptivedynamics._popdyn_model._bindings
	print '_early_bindings:', self._adaptivedynamics._early_bindings
	print '_late_bindings:', self._adaptivedynamics._late_bindings

	if a_name_indexer is None: a_name_indexer = indexer_2d(a_name)
	if r_name_indexer is None: r_name_indexer = indexer(r_name)
	self._a_name_indexer = a_name_indexer
	self._r_name_indexer = r_name_indexer

	# make LV model here
	print 'make LV model'
	self._lv_model = DerivedLotkaVolterraModel( ad._popdyn_model, r_indexer=r_name_indexer, a_indexer=a_name_indexer )

	print '_A_bindings:', self._lv_model._A_bindings

	# for now, this only supports AD with a single phenotype variable
	u_indexers = self._adaptivedynamics._phenotype_indexers

	if False:
	    # sanity check the a values' functional form
	    # using extended system, in case it's only 1-d otherwise
	    extended_system = deepcopy( ad._popdyn_model )
	    # TODO: mutate all populations, not just the first one
	    mutant_index = extended_system.mutate( extended_system._population_indices[0] )
	    extended_lv_model = DerivedLotkaVolterraModel( extended_system, r_name=r_name, a_name=a_name )
	    a_to_u_bindings = self._adaptivedynamics._bindings + self._adaptivedynamics._late_bindings + extended_lv_model._A_bindings
	    print 'a_to_u_bindings:', a_to_u_bindings
            u0s = column_vector( [ u_indexer[0] for u_indexer in u_indexers ] )
            uqis = column_vector( [ u_indexer[mutant_index] for u_indexer in u_indexers ] )
            a_func = a_to_u_bindings( extended_lv_model._indexers['a'][0][mutant_index] ).function( *(list(u0s) + list(uqis)) )
	    print 'a_func:', a_func
	    r_func = a_to_u_bindings( extended_lv_model._indexers['r'][0] ).function( *u0s )
	    print 'r_func:', r_func
	    for i in ad._popdyn_model._population_indices:
                uis = column_vector( u[i] for u in u_indexers )
	        for j in ad._popdyn_model._population_indices:
                    ujs = column_vector( u[j] for u in u_indexers )
	            if a_to_u_bindings( extended_lv_model._indexers['a'][i][j] ) != a_func( *(list(uis)+list(ujs)) ):
		        raise LotkaVolterraException(
			    'Model does not have consistent form a(u_i,u_j): ' +
			    str( a_to_u_bindings( extended_lv_model._indexers['a'][i][j] ) ) +
			    ' does not have the form of ' +
                            'a(%s, %s) = ' % (''.join(str(u[i]) for u in u_indexers), ''.join(str(u[j]) for u in u_indexers)) +
			    str( a_func( *(list(uis) + list(ujs)) ) ) )
		    if a_to_u_bindings( extended_lv_model._indexers['r'][i] ) != r_func( *uis ):
		        raise LotkaVolterraException(
			    'Model does not have consistent form r(u_i): ' +
			    str( a_to_u_bindings( extended_lv_model._indexers['r'][i] ) ) +
			    ' does not have the form of ' +
			    str( r_func( *uis ) ) )
	            for k in ad._popdyn_model._population_indices:
		        if len( extended_lv_model._A_bindings( extended_lv_model._indexers['a'][i][j] ).coefficients( extended_lv_model._population_indexer[k] ) ) > 1:
		            raise LotkaVolterraException( 'Model has super-quadratic terms in population variables' )


	print 'make LV adaptive dynamics'
	print 'population vars', self._lv_model.population_vars()
	print 'population vars', self._lv_model._original_model.population_vars()
        formal_equilibrium = Bindings( { v : hat(v) for v in self._lv_model.population_vars() } )
	self._lv_adap = AdaptiveDynamicsModel(
	    self._lv_model,
	    [ self._lv_model._indexers['r'] ] + [ backward_curry_indexer_2d(self._lv_model._indexers['a'])[j] for j in self._lv_model._population_indices ],
            equilibrium = formal_equilibrium )
	# todo: what to do with multiple phenotype indexers?
	print 'make LV evolution bindings'
	#ui = self._adaptivedynamics._phenotype_indexers[0]
        us = self._adaptivedynamics._phenotype_indexers
	# _A_to_function_bindings expands the LV equations
	# into functions of u_i, e.g. from a_i_j to a(u_i,u_j)
	# very necessary so that partial derivatives can be taken
	from sage.symbolic.function_factory import function
	_A_to_function_dict = [
	  ( a_name_indexer[i][j], 
	    self._lv_model._A_bindings( function( str( a_name_indexer[i][j] ) )( *([u[i] for u in us] + [u[j] for u in us] ) ) ) )
	  for i in self._lv_model._population_indices
	  for j in self._lv_model._population_indices ] + [
	  ( r_name_indexer[i],
	    self._lv_model._A_bindings( function( str( r_name_indexer[i] ) )( *(u[i] for u in us) ) ) )
	  for i in self._lv_model._population_indices ]
	print '_A_to_function_dict:', _A_to_function_dict
	self._A_to_function_bindings = Bindings( dict( _A_to_function_dict ) )

	A_late_bindings = self._lv_model._A_bindings + self._adaptivedynamics._bindings + self._adaptivedynamics._late_bindings
	self._A_function_expansion_dict = dict( [
	    ( a_name_indexer[i][j],
		A_late_bindings( a_name_indexer[i][j] )
		    .function( *([ u_indexer[i] for u_indexer in u_indexers ] + [ u_indexer[j] for u_indexer in u_indexers ]) ) )
		for i in self._lv_model._population_indices
		for j in self._lv_model._population_indices
	] + [
	    ( r_name_indexer[i],
		A_late_bindings( r_name_indexer[i] )
		    .function( *([ u_indexer[i] for u_indexer in u_indexers ]) ) )
		for i in self._lv_model._population_indices
	] )
	print '_A_function_expansion_dict:', self._A_function_expansion_dict
	self._A_function_expansion_bindings = Bindings( FunctionBindings(
	    self._A_function_expansion_dict ) )
	print '_A_function_expansion_bindings:', self._A_function_expansion_bindings
	# _phenotypes_to_fn_bindings: changes u_i to u_i(t)
	# is used in all the below methods, so that du_i(t)/dt works
	self._phenotypes_to_fn_bindings = Bindings( dict(
	    ( u[i], function( str(u[i]), self._adaptivedynamics._time_variable, latex_name=latex(u[i]) ) )
	    for i in self._lv_model._population_indices
	    for u in self._adaptivedynamics._phenotype_indexers ) )
	# _phenotypes_from_fn_bindings: the inverse, change u_i(t) to u_i
	self._phenotypes_from_fn_bindings = Bindings(
	    { v:k for k, v in self._phenotypes_to_fn_bindings.items() } )
	print '_phenotypes_from_fn_bindings:', self._phenotypes_from_fn_bindings
    def A( self, i ):
        '''The 'interaction phenotype' vector of Lotka-Volterra coefficients
        that are affected by selection on population i.  These are r_i and a_ij
        for all j.'''
        # this is a vector, intended to be treated as a column vector.
        return vector(
            [ self._lv_model._indexers['r'][i] ] +
            [ self._lv_model._indexers['a'][i][j] for j in self._lv_model._population_indices ] )
    def S( self, A ):
        '''The selection gradient corresponding to the 'interaction phenotype'
        A.'''
        # This is a vector, intended to be treated as a column vector.
        S = vector( [ self._lv_adap._S[a] for a in A ] )
	#print 'S:', S
	return S
    def d1A( self, i ):
	'''"Direct effect": derivative of A(u_i) wrt u_i with u_i in the "patient"
        position, as the first argument of a(.,.).'''
        # this is a matrix, with row indices the same as in A and column
        # indices indexing the entries of the u phenotype vector.
        A_ij = self._A_to_function_bindings( self.A(i) )
	uis = column_vector( [ u[i] for u in self._adaptivedynamics._phenotype_indexers ] )
	xx = self._adaptivedynamics.fake_population_index()
	uxs = column_vector( [ u[xx] for u in self._adaptivedynamics._phenotype_indexers ] )
	#print 'hack A_ij: from', A_ij
        A_ij_hacked = A_ij.apply_map( lambda x: x.subs_expr( function( str( self._a_name_indexer[i][i] ) )(*(list(uis)+list(uis))) == function( str( self._a_name_indexer[i][i] ) )(*(list(uis)+list(uxs))) ) )
	#print 'to', A_ij_hacked
	#print 'd1A: derivative of $%s$ wrt $%s$' % (latex(A_ij_hacked), latex(uis))
        d1a = matrix( [ [ aij.derivative( u ) for u in uis ] for aij in A_ij_hacked ] )
	#print 'is $%s$' % latex( d1a )
        d = d1a
        for ux, ui in zip( uxs, uis ):
	    d = d.subs( ux == ui )
	#print '; $%s$' % latex( d )
	d = self._A_function_expansion_bindings( d )
	#print ': $%s$' % latex( d )
	return d
    def direct_effect( self, i ):
        # the component of change in A due to direct selection on population i
        # a vector
	return ( self._phenotypes_to_fn_bindings( self.d1A(i) ) * derivative(
	        self._phenotypes_to_fn_bindings( vector( [ u[i] for u in self._adaptivedynamics._phenotype_indexers ] ) ),
	        self._adaptivedynamics._time_variable
	    ) )
    def d2A_component( self, i, j ):
        '''Component of "Indirect effects" on "patient" u_i, due to "agent" u_j.
        When j is i, this includes only the "agent" role of u_i, which is where
        it appears as the second argument of a().'''
        # this is a matrix, with row indices the same as in A and column
        # indices indexing the entries of the u phenotype vector.
        A_ij = self._A_to_function_bindings( self.A(i) )
	u = self._adaptivedynamics._phenotype_indexers[0]
        if j == i:
	    uis = [ u[i] for u in self._adaptivedynamics._phenotype_indexers ]
	    xx = self._adaptivedynamics.fake_population_index()
	    uxs = [ u[xx] for u in self._adaptivedynamics._phenotype_indexers ]
            A_ij_hacked = A_ij.apply_map( lambda x: x.subs_expr( function( str( self._a_name_indexer[i][i] ) )( *(uis+uis) ) == function( str( self._a_name_indexer[i][i] ) )( *(uis+uxs) ) ) )
            #print A_ij,', with', function( self._lv_model._a_name )( *(uis+uis) ), ' => ', function( self._lv_model._a_name )( *(uis+uxs) ), ' == ', A_ij_hacked
	    #print 'd2A: derivative of $%s$ wrt $%s$' %(latex(A_ij_hacked), latex(uxs))
            d2A_hacked = matrix( [ [ aij.derivative( uxi ) for uxi in uxs ] for aij in A_ij_hacked ] )
	    #print 'is $%s$' % latex(d2A_hacked)
            d2A = Bindings( { uxi:uii for uxi, uii in zip(uxs, uis) } )( d2A_hacked )
            #print 'on diagonal: $\partial_2 A = %s$\n\n' % latex( d2A )
        else:
	    #print 'd2A: derivative of $%s$ wrt $%s$' %(latex(A_ij), latex(u[j]))
            d2A = matrix( [ [ aij.derivative( u[j] ) for u in self._adaptivedynamics._phenotype_indexers ] for aij in A_ij ] )
	    #print 'is $%s$'% latex( d2A )
            #ltx.write( 'off diagonal: $\partial_2 A = %s$\n\n' % latex( d2A ) )
	return self._A_function_expansion_bindings( d2A )
    def indirect_effect( self, i ):
        # component of the motion of A_i due to selection on all "agents" that
        # act on population i
        # a vector
	components = [ self._phenotypes_to_fn_bindings( self.d2A_component( i, j ) ) * derivative(
	        self._phenotypes_to_fn_bindings( vector( [ u[j] for u in self._adaptivedynamics._phenotype_indexers ] ) ),
	        self._adaptivedynamics._time_variable
	    ) for j in self._lv_model._population_indices ]
	#import operator
	return reduce( operator.add, components )
    def dudt_bindings( self ):
	# TODO: AD class should provide one of these, and
	# use it internally as well
        # TODO: make matrix * column_vector work
	dudts = {
            i:( SR('gamma') * hat( self._lv_model._population_indexer[i] ) *
	        self.d1A(i).transpose() * self.S( self.A(i) ) )
              for i in self._lv_model._population_indices
        }
        #print 'dudts:', dudts
	return Bindings( {
	    derivative(
                self._phenotypes_to_fn_bindings( us[i] ),
                self._adaptivedynamics._time_variable
            ): dudts[i][j]
	      for i in self._lv_model._population_indices
              for j,us in enumerate(self._adaptivedynamics._phenotype_indexers)
        } )
    def dAdt( self, i ):
        # the motion of the 'interaction phenotype' A_i.  This is not computed
        # using the above components, but by using the chain rule with the
        # derivative of the constraint phenotype u_i.
        # A vector.
        Ai = self.A(i)
	Au = self._adaptivedynamics._bindings( self._lv_model._A_bindings( Ai ) )
	#import operator
        dAdus = [ self._phenotypes_to_fn_bindings( diff( Au, u[j] ) ) for u in self._adaptivedynamics._phenotype_indexers for j in self._lv_model._population_indices ]
        dudts = [ diff( self._phenotypes_to_fn_bindings( u[j] ), self._adaptivedynamics._time_variable ) for u in self._adaptivedynamics._phenotype_indexers for j in self._lv_model._population_indices ]
        dAdts = [ vector( dAiduj * dujdt for dAiduj in dAduj ) for dAduj, dujdt in zip( dAdus, dudts ) ]
	d = reduce( operator.add, dAdts )
	#print 'dAdt: diff of Au is $%s$' % latex( d )
        sys.stdout.flush()
	return d
    def A_index( self, j ):
	# help for using the above vectors: e.g. self.A(i)[self.A_index(j)]
	# is the a_ij term.
	try: self._A_index_dict.keys()
	except AttributeError:
	    self._A_index_dict = dict( (v,i) for i,v in enumerate( [0] + self._lv_model._population_indices ) )
	return self._A_index_dict[j]
    def A_pair( self, i, j ):
	return vector( [
	    self.A(i)[self.A_index(j)],
	    self.A(j)[self.A_index(i)]
	] )
    def S_pair( self, i, j ):
	return vector( [
	    self.S( self.A(i) )[self.A_index(j)],
	    self.S( self.A(j) )[self.A_index(i)]
	] )
    def D_pair( self, i, j ):
	return vector( [
	    self.direct_effect( i )[self.A_index(j)],
	    self.direct_effect( j )[self.A_index(i)]
	] )
    def I_pair( self, i, j ):
	return vector( [
	    self.indirect_effect( i )[self.A_index(j)],
	    self.indirect_effect( j )[self.A_index(i)]
	] )
    def dAdt_pair( self, i, j ):
	return vector( [
	    self.dAdt( i )[self.A_index(j)],
	    self.dAdt( j )[self.A_index(i)]
	] )

def plot_aij_with_arrows( evol_trajectory, lvad, filename=None, scale=1, bindings=Bindings(), **options ):
    resolve_A_bindings = (
	lvad._A_to_function_bindings +
	lvad._A_function_expansion_bindings +
	lvad.dudt_bindings() +
	lvad._phenotypes_from_fn_bindings
    )
    aap = Graphics()
    for i in lvad._lv_model._population_indices:
        for j in lvad._lv_model._population_indices:
	    aap += evol_trajectory.plot(
	        lvad._lv_model._A_bindings( lvad._lv_model._a_indexer[i][j] ),
	        lvad._lv_model._A_bindings( lvad._lv_model._a_indexer[j][i] ),
	        color='red', **options
	    )
	    # plot 4 arrows per (i,j) point
	    pp0 = evol_trajectory._timeseries[0] + bindings + lvad._adaptivedynamics._bindings
	    Aij = (
	        pp0( resolve_A_bindings( lvad.A(i)[lvad.A_index(j)] ) ),
	        pp0( resolve_A_bindings( lvad.A(j)[lvad.A_index(i)] ) )
	    )
	    print 'A:', Aij
	    S = (
	        pp0( resolve_A_bindings( lvad.S( lvad.A(i) )[lvad.A_index(j)] ) ),
	        pp0( resolve_A_bindings( lvad.S( lvad.A(j) )[lvad.A_index(i)] ) )
	    )
	    print 'S:', S
	    if S[0] != 0 or S[1] != 0:
	        aap += arrow( Aij, ( a+scale*s for a,s in zip(Aij, S) ), color='red' )
	    D = (
	        resolve_A_bindings( lvad.direct_effect( i )[lvad.A_index(j)] ),
	        resolve_A_bindings( lvad.direct_effect( j )[lvad.A_index(i)] )
	    )
	    print 'D:', D
	    D = (
	        pp0( D[0] ),
	        pp0( D[1] )
	    )
	    print 'D:', D
	    if D[0] != 0 or D[1] != 0:
	        aap += arrow( Aij, ( a+scale*d for a,d in zip(Aij, D) ), color='green' )
	    I = (
	        resolve_A_bindings( lvad.indirect_effect( i )[lvad.A_index(j)] ),
	        resolve_A_bindings( lvad.indirect_effect( j )[lvad.A_index(i)] )
	    )
	    print 'I:', I
	    I = (
	        pp0( I[0] ),
	        pp0( I[1] )
	    )
	    print 'I:', I
	    if I[0] != 0 or I[1] != 0:
	        aap += arrow( Aij, ( a+scale*i for a,i in zip(Aij, I) ), color='purple' )
	    d = (
	        pp0( resolve_A_bindings( lvad.dAdt( i )[lvad.A_index(j)] ) ),
	        pp0( resolve_A_bindings( lvad.dAdt( j )[lvad.A_index(i)] ) )
	    )
	    print 'dAdt:', d
	    if d[0] != 0 or d[1] != 0:
	        aap += arrow( Aij, ( a+scale*d for a,d in zip(Aij, d) ), color='blue' )
    if filename is not None:
        aap.save( filename, figsize=(5,5), **options )
    return aap
\end{lstlisting}
