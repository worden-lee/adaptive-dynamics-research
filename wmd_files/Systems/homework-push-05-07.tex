\documentclass{exam}
\usepackage{amsmath}
\usepackage[margin=2cm]{geometry}
\usepackage[colorlinks=false]{hyperref}
\usepackage{microtype}
\usepackage{graphicx}
\usepackage{wrapfig}
\usepackage{caption}
\usepackage{subcaption}

\usepackage{pgfplots}
\usepackage{tikz}
\usetikzlibrary{arrows, positioning, calc}

\renewcommand{\d}[1]{\,\mathrm{d}#1}
\newcommand{\e}{\mathrm{e}}

% For "compressed" exam -- no vfills, pagebreaks, etc -- uncomment
% below. The actual value is unimportant; it only matters that it's
% defined. You can also call latex with something like
%
% pdflatex '\newcommand{\compressedformat}{x}\input{exam1.tex}'
%
%\newcommand{\compressedformat}{x}

%
% In the body of the exam, use \myvfill, \mynewpage, and \mycompress
% (the latter is intended to wrap tikzpicture environments, but works
% with anything.)
%
\newcommand{\mycompress}[1]{\ifdefined\compressedformat\relax\else#1\fi}
\newcommand{\myvfill}{\mycompress{\vfill}}
\newcommand{\mynewpage}{\mycompress{\newpage}}

%
% to print a big "DRAFT" on each page, uncomment below
%
%\newcommand{\draftversion}{x}
\newcommand{\drafttext}{\ifdefined\draftversion
  \begin{tikzpicture}[remember picture,overlay]
  \node [rotate=60,scale=12,color=gray!25] at (current page.center)
  {\textsf{DRAFT}};
\end{tikzpicture}
\else
  \relax
\fi}

%
% use this to add "problem continues..."
%
\newcommand{\problemcontinues}{
  \ifdefined\compressedformat
    \relax
  \else
    \runningfooter{}{\thepage\ of \numpages}{problem continues\ldots}
    \newpage
    \runningfooter{}{\thepage\ of \numpages}{}
  \fi}

%
% Change these appropriately.
%
\newcommand{\thedate}{April 30, 2015}
\newcommand{\theexam}{Extra Homework due May 7, 2015}
\newcommand{\thecourse}{Systems, Networks, and Strategies}

\begin{document}

\pagestyle{headandfoot}
\header{\thecourse}{\theexam}{\thedate\drafttext}
\runningheadrule
\firstpageheader{\drafttext}{}{}
\firstpagefooter{}{}{}
\runningfooter{}{\thepage\ of \numpages}{}

\addpoints
\parindent 0ex

\textbf{\thecourse} \hfill  
\textbf{Name:} \makebox[6cm]{\hrulefill}

\textbf{\theexam}

\rule[1ex]{\textwidth}{.1pt}

%%% == Begin ==

\textbf{1. Climate threshold.}
\\*[12pt]
The overall temperature on Earth is affected by several positive and negative
feedback loops that can have complex effects.  For example, rising temperature
causes more water to evaporate, which adds to the greenhouse effect that
captures heat from the sun and raises the temperature more.
Melting ice exposes dark-colored rock, which also captures more heat and raises
the temperature. But raising temperature also causes more heat to radiate
away from the earth, which keeps it from rising forever.  
\\*[12pt]
So when the planet receives a changing amount of solar heat (which it does
all the time), the response of temperature may be complex.  One simplified
model of the global climate gives the relationship between total energy
received from the sun, $Q$, and temperature, $T$, as
\[ Q = \frac{A+BT}{c_i+\frac12 (c_f-c_i)(1+\tanh(\gamma T))} \]
where $A=218$ is the amount of solar energy corresponding to
$T=0^\circ\text{C}$;
$B=1.9$ is the amount of solar energy it takes to raise the temperature
by 1 degree C; $c_i=0.35$ expresses how fast ice soaks up solar energy and
$c_f=0.7$ expresses how fast bare earth soaks up solar energy (faster because
it's darker); and $\gamma$ is a number that summarizes how fast the amount
of ice shrinks when temperature rises.  For more info about this model
see
https://johncarlosbaez.wordpress.com/2012/11/06/mathematics-and-the-environment-part-5/.
\\*[6pt]
\begin{parts}
\part It's hard to find $T$ given $Q$, which is the natural thing to want to
do, but we can plot $Q$ given $T$ and turn it sideways.  Use an online tool or
graphing calculator or whatever to get a plot of this curve, and sketch it 
here or attach a screenshot or printout.  Use $\gamma=0.05$.
\\*[24pt]
\begin{center}
\begin{tikzpicture}
    \begin{axis}[
        width=10cm, height=10cm,
        axis lines=left,           % omit right and top border
        grid = major,
        grid style={dashed, gray!30},
        xmin=0,     % start the diagram at this x-coordinate
        xmax=510,   % end   the diagram at this x-coordinate
        ymin=-53    % start the diagram at this y-coordinate
        ymax=53,    % end   the diagram at this y-coordinate
        /pgfplots/xtick={0,100,...,500},      % draw tick and line every 1
        /pgfplots/ytick={-50,-25,...,50},
        %minor y tick num=9,
        axis background/.style={fill=white},
        ylabel=$T$,
        xlabel=$Q$,
        tick align=inside]

      % mark initial point
      %\addplot [black, mark=*] coordinates {(0, 0)};
    \end{axis} 
\end{tikzpicture}
\end{center}
TODO: fix ticks
\mynewpage

\part That plot shows that for some amounts of incoming solar energy ($Q$),
there is only one fixed-point value of $T$, which predicts what the temperature
will be, but at other values of $Q$, there are three fixed-point values of $T$.
Which value the temperature ends up taking depends on what temperature it
had before. It will be attracted to either the lower or the upper fixed point,
depending on whether it is above or below the middle fixed point. 
If $T$ is near the lower fixed point, it will move to that fixed
point.  Suppose $Q$ is about 400, and $T$ is at about $-40^\circ\mathrm{C}$.
If $Q$ increases gradually to about 450, how will $T$ change?  Sketch it on
the picture and write a brief description here.
\\*[80pt]
\part If $Q$ continues to rise to about 500, what will happen to $T$?  Sketch
it on the picture and describe it here.
\\*[80pt]
\part What will happen to $T$ if $Q$ declines from 500 gradually back to 400
and below?
\\*[80pt]
\part This kind of system behavior is called \emph{hysteresis}, from the
Greek for `memory'. Why might that be?
\\*[80pt]
\part What might be happening in this system, in terms of ice and bare earth?
\\*[80pt]
\end{parts}
\mynewpage

\textbf{2. The doubling map.}
\\*[12pt]
I showed the beginning of this video about the Lorenz attractor and chaos in
class: \href{https://youtu.be/aAJkLh76QnM}{https://youtu.be/aAJkLh76QnM}.
Now watch the rest.
\\*[12pt]
Let's do some work with this doubling model:
\[
  x \mapsto \begin{cases}2x&\text{if }2x<1 \\ 2x-1&\text{if }1\leq2x\end{cases}	
\]
That is, at every step, double $x$, and if the result is 1 or greater, subtract 1.
The video explains how this is a simpler model that captures the key qualities
of the Lorenz model.
\\*[12pt]
If our initial condition is $x=\frac13$, the next $x$ is found by doubling,
giving us $x=\frac23$, and the next $x$ after that is $\frac43-1=\frac13$.
After that $x$ will repeat those two values forever.
\\*[12pt]
If the initial condition is $x=0.2$, as in the video, the next values are
$0.4$, $0.8$, $0.6$, and then it will repeat those four values forever.
\\*[12pt]
\begin{parts}
\part What happens if the initial value is $\frac17$?
\\*[80pt]
\part What happens if the initial value is $\frac12$?  Notice that if doubling
gives 1 exactly, we subtract 1 to get a next value of 0, not 1.
\\*[80pt]
\part If we write `L' when $x$ is smaller than $\frac12$ and `R' when it is 
$\frac12$ or greater, for any initial $x$ we get a sequence of L and R 
describing the system's behavior.  For example, starting with $x=\frac13$
we get LRLRLR$\ldots$, continuing forever, and starting with $x=0.2$ it's
LLRRLLRRLLRR$\ldots$ forever.
\\*[12pt]
\part What sequence do we get starting with $x=\frac12$? Notice that the letter
corresponding to $\frac12$ itself is R, not L.
\\*[80pt]
\part What sequence do we get starting with $x=\frac14$?
\\*[80pt]
\part What sequence do we get starting with $x=\frac18$?
\\*[80pt]
\part What sequence do we get starting with $x=\frac12 + \frac18$?
\\*[80pt]
\part Can you find an initial $x$ that generates the sequence LLLR followed by
an infinite number of Ls?
\\*[80pt]
\part Can you find an initial $x$ that generates the sequence RLLR followed by
an infinite number of Ls?
\\*[80pt]
\part Can you find an initial $x$ that generates any given sequence I might
name followed by an infinite number of Ls?  How?
\\*[80pt]
\part Can you find an initial $x$ that generates any given sequence I can
describe?  How?
(This question might require a more advanced math background than the others.)
\\*[200pt]
\part Does what you've learned about the relationship between numbers and
the sequences they generate help explain why the model has sensitive dependence
on initial conditions?  If so, please explain.
\\*[200pt]
\end{parts}
\vfill

\end{document}
