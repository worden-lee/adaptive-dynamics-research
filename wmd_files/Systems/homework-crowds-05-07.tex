\documentclass{exam}
\usepackage{amsmath}
\usepackage[margin=2cm]{geometry}
\usepackage[colorlinks=false]{hyperref}
\usepackage{microtype}
\usepackage{graphicx}
\usepackage{wrapfig}
\usepackage{caption}
\usepackage{subcaption}

\usepackage{pgfplots}
\usepackage{tikz}
\usetikzlibrary{arrows, positioning, calc}

\renewcommand{\d}[1]{\,\mathrm{d}#1}
\newcommand{\e}{\mathrm{e}}

% For "compressed" exam -- no vfills, pagebreaks, etc -- uncomment
% below. The actual value is unimportant; it only matters that it's
% defined. You can also call latex with something like
%
% pdflatex '\newcommand{\compressedformat}{x}\input{exam1.tex}'
%
%\newcommand{\compressedformat}{x}

%
% In the body of the exam, use \myvfill, \mynewpage, and \mycompress
% (the latter is intended to wrap tikzpicture environments, but works
% with anything.)
%
\newcommand{\mycompress}[1]{\ifdefined\compressedformat\relax\else#1\fi}
\newcommand{\myvfill}{\mycompress{\vfill}}
\newcommand{\mynewpage}{\mycompress{\newpage}}

%
% to print a big "DRAFT" on each page, uncomment below
%
%\newcommand{\draftversion}{x}
\newcommand{\drafttext}{\ifdefined\draftversion
  \begin{tikzpicture}[remember picture,overlay]
  \node [rotate=60,scale=12,color=gray!25] at (current page.center)
  {\textsf{DRAFT}};
\end{tikzpicture}
\else
  \relax
\fi}

%
% use this to add "problem continues..."
%
\newcommand{\problemcontinues}{
  \ifdefined\compressedformat
    \relax
  \else
    \runningfooter{}{\thepage\ of \numpages}{problem continues\ldots}
    \newpage
    \runningfooter{}{\thepage\ of \numpages}{}
  \fi}

%
% Change these appropriately.
%
\newcommand{\thedate}{April 9, 2015}
\newcommand{\theexam}{Homework due April 23, 2015}
\newcommand{\thecourse}{Systems, Networks, and Strategies}

\begin{document}

\pagestyle{headandfoot}
\header{\thecourse}{\theexam}{\thedate\drafttext}
\runningheadrule
\firstpageheader{\drafttext}{}{}
\firstpagefooter{}{}{}
\runningfooter{}{\thepage\ of \numpages}{}

\addpoints
\parindent 0ex

\textbf{\thecourse} \hfill  
\textbf{Name:} \makebox[6cm]{\hrulefill}

\textbf{\theexam}

\rule[1ex]{\textwidth}{.1pt}

%%% == Begin ==

\textbf{Reading:} Ed Yong, ``How the Science of swarms can help us fight cancer and predict the future,'' \emph{Wired}, March 19, 2013.
\\*[12pt]

\textbf{1. Condorcet's Jury Theorem} was established in 1785 by the Marquis de
Condorcet, a philosopher, mathematician and political scientist who
held several government positions after the French Revolution, who advocated
the abolition of slavery and did research into the dynamics of voting and
democratic practices.
\\*[12pt]
His Jury Theorem concerns a group trying to make a decision by majority vote.
If each person has probability $p$ of voting for the ``correct'' decision,
and their votes are not influenced by each other, and the probability $p$ is
greater than $1/2$, then the probability that the group votes for the
correct decision is greater than $p$. That is, the group is a better decision
maker than each person in the group. Bigger groups are better decision makers,
and if we consider larger and larger groups the probability of the correct
decision approaches 1.
\\*[12pt]
Here we explore why, using this diagram.

% Set the overall layout of the tree
\tikzstyle{level 1}=[level distance=3.5cm, sibling distance=7cm]
\tikzstyle{level 2}=[level distance=3.5cm, sibling distance=3.5cm]
\tikzstyle{level 3}=[level distance=3.5cm, sibling distance=2cm]

% Define styles for bags and leafs
\tikzstyle{bag} = [text width=4em, text centered]
\tikzstyle{end} = [circle, minimum width=3pt,fill, inner sep=0pt]

% http://www.texample.net/tikz/examples/probability-tree/
\begin{center}
% The sloped option gives rotated edge labels. Personally
% I find sloped labels a bit difficult to read. Remove the sloped options
% to get horizontal labels. 
\begin{tikzpicture}[grow=right, sloped]
\node[bag] {Starting from nothing}
    child {
        node[bag] {\textbf{N} $0.4$}
            child {
                node[bag] {\textbf{N, N} $0.16$}
		    child {
			node[bag] {\textbf{N,~N,~N} \begin{minipage}[b][1.5em]{4em}{\hrulefill}\end{minipage}}
			edge from parent
			node[above] {vote N}
			node[below]  {$0.4$}
		    }
		    child {
			node[bag] {\textbf{N,~N,~Y} \begin{minipage}[b][1.5em]{4em}{\hrulefill}\end{minipage}}
			edge from parent
			node[above] {vote Y}
			node[below]  {$0.6$}
		    }
                edge from parent
                node[above] {vote N}
                node[below]  {$0.4$}
            }
            child {
                node[bag] {\textbf{N, Y} $0.24$}
		    child {
			node[bag] {\textbf{N,~Y,~N} \begin{minipage}[b][1.5em]{4em}{\hrulefill}\end{minipage}}
			edge from parent
			node[above] {vote N}
			node[below]  {$0.4$}
		    }
		    child {
			node[bag] {\textbf{N,~Y,~Y} \begin{minipage}[b][1.5em]{4em}{\hrulefill}\end{minipage}}
			edge from parent
			node[above] {vote Y}
			node[below]  {$0.6$}
		    }
                edge from parent
                node[above] {vote Y}
                node[below]  {$0.6$}
            }
            edge from parent 
            node[above] {vote N}
            node[below] {$0.4$}
    }
    child {
        node[bag] {\textbf{Y} $0.6$}
        child {
                node[bag] {\textbf{Y, N} $0.24$}
		    child {
			node[bag] {\textbf{Y,~N,~N} \begin{minipage}[b][1.5em]{4em}{\hrulefill}\end{minipage}}
			edge from parent
			node[above] {vote N}
			node[below]  {$0.4$}
		    }
		    child {
			node[bag] {\textbf{Y,~N,~Y} \begin{minipage}[b][1.5em]{4em}{\hrulefill}\end{minipage}}
			edge from parent
			node[above] {vote Y}
			node[below]  {$0.6$}
		    }
                edge from parent
                node[above] {vote N}
                node[below]  {$0.4$}
            }
            child {
                node[bag] {\textbf{Y, Y} $0.36$}
		    child {
			node[bag] {\textbf{Y,~Y,~N} \begin{minipage}[b][1.5em]{4em}{\hrulefill}\end{minipage}}
			edge from parent
			node[above] {vote N}
			node[below]  {$0.4$}
		    }
		    child {
			node[bag] {\textbf{Y,~Y,~Y} $0.216$}
			edge from parent
			node[above] {vote Y}
			node[below]  {$0.6$}
		    }
                edge from parent
                node[above] {vote Y}
                node[below]  {$0.6$}
            }
        edge from parent         
            node[above] {vote Y}
            node[below] {$0.6$}
    };
\end{tikzpicture}
\end{center}
In this example, we call the ``correct'' decision \textbf{Y}, and we
suppose that $p$ is 0.6, the probability that a given
person chooses \textbf{Y}. So the probability of choosing \textbf{N} is 0.4,
since they must add to 1.
And the probability that the first
person chooses \textbf{Y} and the second person chooses \textbf{Y} is 0.6
times 0.6, which is 0.36.  If you follow along the very top of this diagram
from left to right you'll see that 0.36 there. 

\mynewpage

The probability that the
first three people choose \textbf{Y, Y, Y} is also found by multiplying:
the first two choose \textbf{Y, Y} with probability 0.36, and the third
chooses \textbf{Y} with probability 0.6, and  0.36 times 0.6 is 0.216. That's
the value at the top right.
\\*[6pt]
\begin{parts}
\part Fill in the blanks in that diagram for the other choices that three 
people can make.
\\*[40pt]
\part In a group of two people, the probability of a tie vote is found by
adding: the two ways it can happen are \textbf{Y,~N} and \textbf{N,~Y}.  
The sum of those probabilities is 0.24 + 0.24 = 0.48.
\\*[14pt]
In a group of three people, list the different ways they can come out
to exactly two \textbf{Y} votes and one \textbf{N} vote.
\\*[80pt]
\part Using the numbers you added to the diagram, what is the total probability of getting
two \textbf{Y} votes and one \textbf{N} vote?
\\*[80pt]
\part What is the probability of three \textbf{Y} votes?
\\*[80pt]
\part Adding them together, what is the probability that in a group of three 
people a majority votes for \textbf{Y}?
\\*[80pt]
\end{parts}
\vfill
The probability of a majority vote for \textbf{Y} should come out larger
than 0.6, meaning that the group is more likely to make the ``correct''
decision than any one person.  It becomes \emph{much} more likely when more
people vote.

\mynewpage

\textbf{2. When people don't decide independently}
Unfortunately, Condorcet's theorem only works when people make their choices
\emph{independently}, without influencing each other.  When people influence
each other, some of the benefit of the ``wisdom of the crowd'' can be lost.
\\*[12pt]
Here we look just a little at this situation. As before, suppose each person
prefers the ``correct'' decision, \textbf{Y}, with probability 0.6.  But
before they vote, they look at all the information they know about -- their own
preference, and the people who have already voted -- and if their preference
is outnumbered they vote with the majority.  That is, if your preference is
\textbf{N} and the two people before you voted \textbf{Y}, you go with the
crowd and vote \textbf{Y}. That means you vote \textbf{Y} in that case
regardless of your preference.
\\*[12pt]
Because of that, this diagram is different from the last one
in just a few places. If someone is
outnumbered, they go with the majority no matter what they prefer, so it's
marked ``always'' and it happens with probability 1, that is, they
always make that choice.  So the probability of \textbf{Y, Y, Y} is the
same as the probability of \textbf{Y, Y} because after \textbf{Y, Y} the 
third person always votes \textbf{Y}.  And \textbf{Y, Y, N} never happens,
so its probability is 0.
\\*[12pt]
\begin{center}
% The sloped option gives rotated edge labels. Personally
% I find sloped labels a bit difficult to read. Remove the sloped options
% to get horizontal labels. 
\begin{tikzpicture}[grow=right, sloped]
\node[bag] {Starting from nothing}
    child {
        node[bag] {\textbf{N} $0.4$}
            child {
                node[bag] {\textbf{N, N} $0.16$}
		    child {
			node[bag] {\textbf{N,~N,~N} \begin{minipage}[b][1.5em]{4em}{\hrulefill}\end{minipage}}
			edge from parent
			node[above] {vote N}
			node[below]  {always}
		    }
		    child {
			node[bag] {\textbf{N,~N,~Y} \begin{minipage}[b][1.5em]{4em}{\hrulefill}\end{minipage}}
			edge from parent
			node[above] {vote Y}
			node[below]  {never}
		    }
                edge from parent
                node[above] {vote N}
                node[below]  {$0.4$}
            }
            child {
                node[bag] {\textbf{N, Y} $0.24$}
		    child {
			node[bag] {\textbf{N,~Y,~N} \begin{minipage}[b][1.5em]{4em}{\hrulefill}\end{minipage}}
			edge from parent
			node[above] {vote N}
			node[below]  {$0.4$}
		    }
		    child {
			node[bag] {\textbf{N,~Y,~Y} \begin{minipage}[b][1.5em]{4em}{\hrulefill}\end{minipage}}
			edge from parent
			node[above] {vote Y}
			node[below]  {$0.6$}
		    }
                edge from parent
                node[above] {vote Y}
                node[below]  {$0.6$}
            }
            edge from parent 
            node[above] {vote N}
            node[below] {$0.4$}
    }
    child {
        node[bag] {\textbf{Y} $0.6$}
        child {
                node[bag] {\textbf{Y, N} $0.24$}
		    child {
			node[bag] {\textbf{Y,~N,~N} \begin{minipage}[b][1.5em]{4em}{\hrulefill}\end{minipage}}
			edge from parent
			node[above] {vote N}
			node[below]  {$0.4$}
		    }
		    child {
			node[bag] {\textbf{Y,~N,~Y} \begin{minipage}[b][1.5em]{4em}{\hrulefill}\end{minipage}}
			edge from parent
			node[above] {vote Y}
			node[below]  {$0.6$}
		    }
                edge from parent
                node[above] {vote N}
                node[below]  {$0.4$}
            }
            child {
                node[bag] {\textbf{Y, Y} $0.36$}
		    child {
			node[bag] {\textbf{Y,~Y,~N} 0}
			edge from parent
			node[above] {vote N}
			node[below]  {never}
		    }
		    child {
			node[bag] {\textbf{Y,~Y,~Y} $0.36$}
			edge from parent
			node[above] {vote Y}
			node[below]  {always}
		    }
                edge from parent
                node[above] {vote Y}
                node[below]  {$0.6$}
            }
        edge from parent         
            node[above] {vote Y}
            node[below] {$0.6$}
    };
\end{tikzpicture}
\end{center}

\mynewpage
\begin{parts}
\part Fill in the blanks in that diagram.
\\*[40pt]
\part If the first two people vote \textbf{N}, what does the third person do?
\\*[80pt]
\part If the first two people voted \textbf{N}, given what the third person
does, what will the fourth person do?
\\*[80pt]
\part And after that, what will the fifth and sixth person do?
\\*[80pt]
\part If the first four people vote \textbf{N, Y, N, N}, what will the fifth person do?
\\*[80pt]
\part In these cases, will the majority vote for \textbf{Y} or \textbf{N}?
\end{parts}
\vfill
This is called an \emph{opinion cascade} or \emph{information cascade}. Once
a cascade starts, a lot of people's insight is unused, and the group can no
longer be trusted to make the right choice, even if it's a large group.  It's
worth watching
out for opinion cascades -- sometimes when a lot of people are making the same
choice, it's actually just because they're mostly following along,
not because it's a good idea.

\end{document}
